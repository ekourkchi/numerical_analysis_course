% !TEX TS-program = latex
\documentclass[12pt]{article}


\begin{document}
\setlength\parindent{0cm}
%\thispagestyle{empty}

Astr 735 -- Fall 2015 -- Scientific Computing -- Norbert Schorghofer\\


{\bf Homework \#1}

Survey on Google Forms\\

{\bf Homework \#2}

a) Plot the function  $f(x) = 3 \pi^4 x^2+\ln((x-\pi)^2)$ in the range 0 to 4.\\   % with a graphing software of your choice.
b) Proof that $f(x)$ has two roots, $f(x)=0$.\\
c) Estimate the distance between the two roots. \\
d) Plot the function in the vicinity of these roots.  Does the function change sign, as it must?\\

%{\it Solution:} The graph of the function appears to be positive everywhere, but it's not. At $x=\pi$, $f(x) = -\infty$.  Hence, the function has two roots very close to $x=\pi$. It has no additional roots, as $f$ is clearly positive far from $\pi$. With $x=\pi+\epsilon$, $\ln\epsilon^2 \approx -3 \pi^6$, and thus $|\epsilon| \approx e^{-1442} = e^{1442/\ln 10} \approx 10^{-626}$.  Your plotting program does not resolve distances this small, no matter how short a range one picks.\\


{\bf Homework \#3}

\begin{enumerate}
\item
Using a programming language of your choice, find out whether
$1.2-1-0.2==0$ is true or false. Also state the language you used.

\item
The iteration $x_{n+1} = 4x_n (1-x_n)$ is extremely sensitive to the initial value as well as to roundoff. Yet, thanks to the IEEE 754 standard, it is possible to reproduce the exact same sequence on many platforms.

Calculate the first 1,010 iterations with initial conditions $x_0=0.8$ and $x_0 = 0.8+10^{-15}$. Use double (8-byte) precision.  
If IEEE 754 compliance is not enabled by default for the tool/language you use, find out how to enable it. 
Submit the program as well as the values $x_{1000} ... x_{1009}$ to 10 digits after the comma.
We will compare the results in class.

\end{enumerate}

%Everyone should get the same result, irrespective of programming language, compiler, operating system, and type of CPU.  

%{\it Solution 1}: Students used five different programming languages (Python, C, C++, Matlab, and Lisp) and all obtained \texttt{False}. Comment: $1.2+(-1-0.2)==0$ would be \texttt{True}.  Lesson: When making a comparison with a floating point number one {\it must} allow for a tolerance.

%Solution 2: See slides\\
% Note that languages have compiler options

{\bf Homework \#4}

Practice your programming language or learn a new one or get acquainted with an editor like emacs or vi. No need to turn in anything.\\


{\bf Homework \#5}

For the kicked rotator ($\alpha_{n+1}=\alpha_n+\omega_n T$, $\omega_{n+1}=\omega_n + K\sin\alpha_{n+1}$), determine how fast the energy $E=\omega^2/2$ grows with time $n$.  
Consider the ensemble average and large kicking strength ($K \geq 4$). You need to consider only one value of $K$; the answer is supposedly independent of $K$ beyond this value.
Take initial values that lead to chaotic motions, that is, find a way to exclude integrable solutions from the average.
The function you fit should have a physically reasonable asymptotic behavior.

%Take initial values that lead to chaotic motions and large kicking strength ($K \geq 4$).  Consider the ensemble average, but find a way to exclude integrable solutions from the average.\\

Send:  1) a plot of the ensemble-averaged energy as a function of time and a functional fit to this graph, 
2) a description of how you avoided the integrable solutions, and 
3) basic info such as the value of $K$ used, how many initial values were chosen and how they were chosen, and whatnot.\\

%The goal is simple: Find out how fast the energy (of the chaotic solutions) grows with time.



{{\bf Homework \#6}

Solve Kepler's equation with Newton's method.\\

The Kepler equation is
$M=E-e\sin E$, where $M$ is the mean anomaly (linear in time), $E$ the eccentric anomaly, and $e$ the eccentricity.
The distance from the sun is $r = a(1-e\cos E)$.
Write a function that reliably solves this equation for any $M$.
Use a reasonable criterion to decide how many iterations are necessary.
Test the program with exact solutions.
Use $e=0.9671$, appropriate for Halley's comet.

\hspace{0.25in}
Calculate the time average of $(a/r)^2$ to at least three significant digits. The mean solar flux is proportional to this quantity.
(This average can be obtained analytically, but the task here is to obtain it numerically.)\\




{\bf Homework \#7}

The fragment of a Fortran program below finds the distance between the nearest pair among $N$ points in two dimensions.  This implementation is wasteful in computer resources.
Can you suggest at least 4 simple changes that will improve its computational performance?
(You are {\it not} asked to verify that they improve performance.)

\begin{verbatim}
  ! ... x and y have been assigned values earlier ...
  m=1E38
  do i=1,N
     do j=1,N
        if (i==j) cycle  ! skips to next iteration
        r(i,j) = sqrt((x(i)-x(j))**2. + (y(i)-y(j))**2.)
        if (r(i,j)<m) m=r(i,j)
     enddo
  enddo
  ! m is minimum distance
\end{verbatim}

\vspace{1em}
{\bf Homework \#8}

What is the arithmetic intensity for the full gravitational N-body
problem when evaluated with a simple nested double loop?
Steps: 1. Write down program code or pseudocode that calculates the acceleration of each body
in 3D, 2. Count the number of floating point operations required for
the evaluation, to leading order. Assume the square root operation is
equivalent to 10 FLOPs. 3. Count the number of bytes that need to be
accessed from main memory. Assume floating point numbers with 8 bytes. 
4. Take the ratio.
Based on this ratio, do you think this calculation is floating-point
limited or data-transfer limited?
\\




\end{document}


